\documentclass[12pt,a4paper]{article}

% --- Pakiety językowe i kodowanie ---
\usepackage[T1]{fontenc}
\usepackage[utf8]{inputenc}
\usepackage[polish]{babel}

% --- Pakiety graficzne i formatowanie ---
\usepackage{graphicx}
\usepackage{geometry}
\usepackage{hyperref}
\usepackage{enumitem}
\usepackage{xcolor}
\usepackage{float}

% Ustawienia marginesów
\geometry{margin=2.5cm}

% Ustawienia wyglądu linków
\hypersetup{
    colorlinks=true,
    linkcolor=blue,
    urlcolor=cyan,
}

% --- Tytuł i autor ---
\title{Instrukcja instalacji Ubuntu na VirtualBox \\ \large Krok po kroku}
\author{Sviatoslav Kopytin}
\date{\today}

\begin{document}

\maketitle
\tableofcontents
\newpage

\section{Wstęp}
Niniejszy dokument zawiera szczegółową instrukcję konfiguracji środowiska wirtualnego za pomocą programu \textbf{Oracle VM VirtualBox} oraz instalacji systemu operacyjnego \textbf{Ubuntu}.

\section{Wymagania wstępne}
Zanim zaczniesz, upewnij się, że posiadasz:
\begin{itemize}
    \item Pobrany instalator VirtualBox ze strony \href{https://www.virtualbox.org}{virtualbox.org}.
    \item Obraz ISO systemu Ubuntu (wersja Desktop) ze strony \href{https://ubuntu.com/download/desktop}{ubuntu.com}.
    \begin{figure}[htbp]
    \centering
    \includegraphics[width=0.8\textwidth]{Screenshots/scr1.png}
\end{figure}
    \item Minimum 4GB pamięci RAM oraz 25GB wolnego miejsca na dysku.
\end{itemize}

\section{Konfiguracja Maszyny Wirtualnej}

\subsection{Krok 1: Tworzenie nowej maszyny}
Otwórz VirtualBox i kliknij przycisk \textbf{Nowa (New)}.


\begin{enumerate}
    \item Podaj nazwę maszyny (np. "Ubuntu\_24").
    \item Wybierz folder, w którym maszyna ma zostać zapisana.
    \item Wskaż pobrany wcześniej obraz ISO.
\end{enumerate}

\begin{figure}[H]
    \centering
    \includegraphics[width=0.8\textwidth]{Screenshots/scr2.png}
    \includegraphics[width=0.8\textwidth]{Screenshots/scr3.png}
\end{figure}

\subsection{Krok 2: Przydzielenie zasobów}
Dla płynnego działania zaleca się ustawienie:
\begin{itemize}
    \item \textbf{Procesor:} Minimum 2 rdzenie.
    \item \textbf{Pamięć operacyjna (Base Memory):} Minimum 2048 MB (zalecane 4096 MB).
    \item \textbf{Dysk (EFI):} Zaznacz "Enable EFI" (opcjonalnie, zależnie od wersji).
\end{itemize}

\begin{figure}[H]
    \centering
    \includegraphics[width=0.8\textwidth]{Screenshots/scr4.png}
\end{figure}

\section{Uruchamianie maszyny wirtualnej}
Po uruchomieniu maszyny (przycisk \textbf{Start}), należy poczekać.
Pierwsze uruchomienie maszyny może potrwać dosyć długo, więc należy poczekać.

\begin{figure}[H]
    \centering
    ...Uruchamianie maszyny, poniżej jeden ze slajdów.
    \includegraphics[width=0.8\textwidth]{Screenshots/scr5.png}
\end{figure}


\section{Konfiguracja poinstalacyjna (opcjonalna)}
Po zakończeniu instalacji warto zainstalować \textit{Guest Additions}, aby odblokować współdzielenie schowka i pełny ekran. W terminalu Ubuntu wykonaj:

\begin{verbatim}
sudo apt update
sudo apt upgrade
sudo apt install build-essential dkms linux-headers-$(uname -r)
\end{verbatim}

\begin{figure}[H]
    \centering
    \includegraphics[width=0.8\textwidth]{Screenshots/scr6.png}
\end{figure}



\section{Podsumowanie}
Twoja maszyna wirtualna jest gotowa do pracy. Możesz teraz bezpiecznie testować oprogramowanie wewnątrz odizolowanego systemu.
:)

\end{document}